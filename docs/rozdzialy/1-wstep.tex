\section{Wstęp}
    \subsection{Wprowadzenie}
    Podróżując po dużych miejscowościach i miastach, zarówno tych w kraju ojczystym jak i podczas wycieczek za granicą, osobiście brakowało mi możliwości sprawdzenia jakie atrakcje turystyczne czekają na mnie w odwiedzanych przeze mnie miastach a także sprawdzenia tego, co wartego uwagi znajduje się w danym momencie w pobliżu. Bardzo często zdarzało się omijać interesujące zabytki lub muzea o których istnieniu nie miałem pojęcia podczas pobytu w danym mieście. Dowiadywałem się o nich dopiero po powrocie z podróży do domu. Możliwym rozwiązaniem tego problemu jest szukanie informacji o tym co można zobaczyć w miejscu do którego mam plany się udać w podróż na długo przed samym wyruszeniem w drogę, zajmowało to jednak bardzo dużo czasu gdyż takie informacje znajdowały się na wielu portalach internetowych i konieczne było przeglądanie wielu źródeł różnej jakości co wymagało ich weryfikacji. Nie zawsze też takie plany dochodziły do skutku, przez co poświęcałem dużo czasu na planowaniu wycieczki która nigdy się nie odbyła.
    
    Z wyżej wspomnianych powodów postanowiłem stworzyć aplikację aby rozwiązać te problemy. Celem tego programu jest przede wszystkim oferowanie pomocy w tworzeniu planów wycieczek i w rezultacie oszczędzenie sporej ilości czasu. Głównym założeniem jest posiadanie wszystkich informacji o atrakcjach turystycznych w jednym miejscu co pozwoli na szybkie przeszukiwanie i sprawdzanie potencjalnie interesujących nas atrakcji w miejscu do którego chcemy się w najbliższym czasie udać.
    
    Warto jednak pamiętać że zupełnie innym doświadczeniem jest zwiedzanie obcej kultury samotnie, a czym innym w grupie. Zwiedzanie zabytków i muzeów jest dobrym miejscem aby poznać nowych ludzi którzy przybyli w tym samym celu co my, czyli odkryć coś nowego, dzięki czemu możemy się wymienić swoimi spostrzeżeniami i poznać punkt widzenia drugiej osoby. W ten sposób jest szansa aby spojrzeć na dany temat z innej strony, takiej której wcześniej nie mogliśmy zauważyć przez co poszerzymy swoje horyzonty. To co może przeszkodzić nam w realizacji tego celu jest nasza skrytość, nieśmiałość bądź też nieufność wobec obcych dla nas osób lub brak możliwości komunikacji z takimi osobami.

    \subsection{Przegląd istniejących rozwiązań}
    Na rynku istnieją aktualnie rozwiązania które rozwiązują część z podanych wcześniej założeń. Poniżej przedstawiono takie aplikacje, które spełniają najwięcej z nich: \\

    \textbf{Meetup} \\ 
    Meetup to aplikacja stworzona do poznawania ludzi z podobnymi zainteresowaniami do naszych. Główną cechą tego rozwiązania jest możliwość dołączania do społeczności osób które organizują wydarzenia w prawdziwym świecie dla wszystkich zainteresowanych członków danej grupy. Dostępne są również wydarzenia publiczne do których może dołączyć każdy, niezależnie od tego czy znajduje się w społeczności organizującej czy też nie. Jeden użytkownik może być członkiem wielu grup. Głównym celem tej aplikacji jest zapoznanie z innymi użytkownikami w prawdziwym świecie. \\
    Aplikacja ta nie skupia się na jednej konkretnej aktywności a zrzesza wielu różnych odbiorców. Z tego powodu nie zawiera udogodnień które są niezbędne dla turystów, w Meetup nie znajdziemy informacji o dostępnych obiektach historycznych ani ich lokalizacji czy podstawowych informacji takich jak godziny otwarcia. \\ 

    \textbf{Nearify} \\
    Nearify pozwala na znajdowanie wydarzeń odbywających się w naszym pobliżu. Możemy wybierać różne rodzaje wydarzeń i mamy również możliwość sprawdzenia kto dokładnie idzie na dane spotkanie. \\

    \textbf{Tripadvisor} \\ 
    Tripadvisor jest bardzo dobrym przewodnikiem dla turystów którzy mają w planach wybrać się w nowe miejsce. Możemy tu sprawdzić atrakcje turystyczne dostępne w danym miejscu, porównywać hotele, przeglądać restauracje i sprawdzać rodzaje serwowanych dań. Nie jest to jednak wszystko, Tripadvisor pozwala również oceniać wszystkie wcześniej wspomniane obiekty w skali od jednej do pięciu gwiazdek. \\
    Istnieje także forum dyskusyjne Tripadvisora, na którym zarejestrowani użytkownicy mogą kontaktować się z innymi użytkownikami. Nie możemy jednak rozmawiać sam na sam z innymi użytkownikami. \\

    \textbf{Travello} \\
    Travello to aplikacja która określa się mianem "medium społecznościowym dla podróżników". Istotnie, w tej aplikacji możemy poznawać innych użytkowników tej aplikacji znajdujących się blisko nas, dzielić się zdjęciami z podróży, dołączać do grup społecznościowych o wspólnych zainteresowaniach do naszych a także, co się może bardzo przydać, znajdować strefy z bezpłatnym dostępem do internetu bezprzewodowego. \\
    Aplikacja skupia się na społecznościowym aspekcie podróżowania. Nie możemy sprawdzić samych dostępnych atrakcji w miejscu które planujemy odwiedzić, musimy w tym celu użyć innej aplikacji. \\ 

    \textbf{UNBLND} \\ 
    Aplikacja ta skupia się przede wszystkim na zapoznaniu się z nowymi osobami. UNBLND dodaje nas do grup na podstawie wcześniej przesłanych zainteresowań. Możliwe jest tworzenie podgrup w danej grupie co pozwala nam na bliższe zapoznanie się z innymi użytkownikami tej aplikacji. \\
    Aspekt turystyczny w rozwiązaniu tym nie istnieje. Nie mamy możliwości sprawdzenia czy w pobliżu możemy zobaczyć obiekt historyczny lub inną instytucję kulturalną. \\

    \textbf{Travel Buddy} \\ 
    Travel Buddy to aplikacja która w głównej mierze skupia się na poznawaniu mieszkańców miejsc do których się udajemy. Dzięki temu, możemy poznać historię i kulturę od środka, czego nie udałoby nam się dokonać poprzez samo oglądanie zabytków. \\ 
    Głównym celem tej aplikacji jest, jak sama nazwa wskazuje, poszukiwanie towarzysza podróży. Mamy możliwość dopasowania naszych poszukiwań pod kątem daty i celu podróży. \\
    Tak jak w UNBLND, aplikacja jest skupiona na poznawaniu ludzi. Nie sprawdzimy tutaj atrakcji które na nas czekają. \\

    \textbf{Worldpackers} \\ 
    Dzięki tej aplikacji, możemy łatwo znaleźć noclegi w miejscu do którego się udajemy. Przy wcześniejszym podaniu umiejętności i zainteresowań, aplikacja za pomocą algorytmu proponuje nam miejsca do noclegu które mogą nam się spodobać. Możemy również wybrać to co chcemy robić na miejscu, do wyboru mamy na przykład naukę języka bądź gotowanie co też jest liczone przy wybieraniu miejsca noclegowego dla nas. \\
    W aplikacji tej działa również forum dyskusyjne, w którym zarejestrowani użytkownicy dają rekomendacje i wskazówki co do podróżowania. Nie ma za to informacji co jest dostępne na miejscu do którego chcemy się udać. \\

    \textbf{Tripwolf} \\ 
    Aplikacja ta pozwala na tworzenie przewodników turystycznych i udostępnianie ich innym użytkownikom. Jest to bardzo dobry sposób aby zapoznać się z tym co nas może czekać podróżując w dane miejsce. Możemy również dzielić się wskazówkami dotyczącymi danego miejsca, chociażby na jakie zabytki warto zwrócić uwagę lub jakie restauracje oferują dobre regionalne jedzenie. Przewodniki są dostępne w kilku językach: angielskim, francuskim, niemieckim, hiszpańskim, włoskim i rosyjskim. Dodatkowe informacje na temat różnych zabytków i miast są pobierane z Wikipedii. \\
    Aplikacja w porównaniu do kilku poprzednich, skupia się jedynie na aspekcie turystycznym. Nie możemy skontaktować się z innymi użytkownikami tej aplikacji.
