\newpage
\section{Projekt}
Celem niniejszej pracy jest stworzenie aplikacji w wersji webowej i na systemy mobilne Android oraz iOS pozwalające na przegląd atrakcji turystycznych w różnych miastach w Polsce, Europie i na świecie. W ramach pracy przewidziane jest:
\begin{enumerate}
    \item pogłębienie wiedzy na temat systemu operacyjnego Android i wytwarzania oprogramowania na tą platformę,
    \item zaprojektowanie prototypu aplikacji dostępnej dla przeglądarek internetowych
    \item zaprojektowanie prototypu aplikacji na urządzenia przenośne
    \item implementacja systemu jako aplikację webową
    \item implementacja systemu na platformę Android,
    \item integracja części prezentacji aplikacji z usługami oferowanymi przez Firebase,
    \item testy aplikacji
\end{enumerate}

    \subsection{Badanie rynku}
    W ramach pracy, przeprowadzone zostało badanie rynku wśród internautów za pomocą anonimowej ankiety. Celem tego badania było rozpoznanie sytuacji i zrozumienie potrzeb docelowych użytkowników takiej aplikacji.

    \subsection{Wymagania}
    Określono główne funkcjonalności aplikacji z podziałem na poszczególne rodzaje wymagań: \\
    \textbf{Wymagania użytkowe dotyczące zarządzania kontem użytkownika}
    \begin{itemize}
        \item założenie konta i profilu użytkownika,
        \item przechowywanie sesji użytkownika na urządzeniu mobilnym,
        \item logowanie do aplikacji przy użyciu istniejącego konta użytkownika,
        \item usunięcie konta i profilu użytkownika,
        \item założenie konta użytkownika przy użyciu istniejącego konta Google,
        \item założenie konta użytkownika przy użyciu istniejącego konta Facebook,
        \item założenie konta użytkownika przy użyciu istniejącego konta X,
        \item edycja profilu użytkownika,
        \item wgrywanie własnego zdjęcia z urządzenia reprezentującego użytkownika w aplikacji,
    \end{itemize}
    \textbf{Wymagania dotyczące przeglądania atrakcji turystycznych}
    %Wymagania funkcjonalne aplikacji zostały przedstawione poniżej:
    \begin{itemize}
        \item przeglądanie listy pobliskich atrakcji turystycznych,
        \item prezentowanie odległości do danej atrakcji turystycznej w metrach bądź kilometrach,
        \item wyświetlanie informacji o danych zabytkach, takich jak lokalizacja, godziny otwarcia i ceny biletów,
        \item prezentowanie zdjęć danej atrakcji turystycznej użytkownikom,
        \item przeglądanie listy wydarzeń kulturalnych organizowanych przez instytucje kulturalne takie jak muzea czy teatry,
        \item dodawanie nowych atrakcji turystycznych przez użytkowników,
        \item dodawanie atrakcji turystycznych do listy ulubionych atrakcji,
        \item usuwanie atrakcji turystycznych z listy ulubionych atrakcji
    \end{itemize}
    \textbf{Wymagania dotyczące elementów społecznościowych}
    \begin{itemize}
        \item przeglądanie listy aktywnych użytkowników,
        \item przeglądanie profilów innych użytkowników,
        \item wysyłanie zaproszeń na wspólne zwiedzanie do innych użytkowników,
        \item zaakceptowanie zaproszenia otrzymanego przez innego użytkownika,
        \item odrzucenie zaproszenia otrzymanego przez innego użytkownika,
        \item zablokowanie możliwość komunikacji z innym użytkownikiem,
        \item zgłoszenie innego użytkownika
    \end{itemize}
    \textbf{Pozostałe wymagania funkcjonalne aplikacji}
    \begin{itemize}
        \item przeglądanie numerów alarmowych dostępnych w danej lokalizacji,
        \item dostępność aplikacji w języku angielskim,
        \item dostępność aplikacji w języku polskim,
        \item korzystanie z aplikacji bez potrzeby zakładania konta użytkownika
    \end{itemize}
    \textbf{Wymagania niefunkcjonalne aplikacji} zestawiono poniżej:
    \begin{itemize}
        \item uruchomienie aplikacji na systemie Android w wersji 12,
        \item uruchomienie aplikacji na systemie iOS,
        \item uruchomienie aplikacji na przeglądarce,
        \item umożliwienie rozwoju aplikacji w wersji natywnej na komputery osobiste,
        \item zapewnienie bezpieczeństwa aplikacji przed atakami z zewnątrz
        \item wykorzystanie dostępnych rozwiązań opartych na usługach chmurowych,
        \item zastosowanie dobrych praktyk programowania
    \end{itemize}

    \subsection{Przypadki użycia}
    \textbf{Rejestracja użytkownika} \\ 
       Opis: Tworzenie konta nowego użytkownika \\
       Aktorzy: Wszyscy, którzy nie posiadają aktywnego konta w aplikacji \\ 
       Warunki wstępne: Aktor nie jest zalogowany i chce utworzyć nowe konto w aplikacji, aby otrzymać dostęp do jej funkcjonalności \\ 
       Podstawowy przebieg:
       \begin{enumerate}
           \item Naciśnięcie przycisku do rejestracji nowego użytkownika
           \item Otworzenie nowego widoku z prośbą o podanie adresu mailowego i nowej nazwy użytkownika
           \item System sprawdza poprawność wprowadzonych danych
           \begin{enumerate}
               \item W przypadku odmowy, system informuje użytkownika o tym zdarzeniu
               \item W przypadku zatwierdzenia danych, użytkownik dostaje informacje o wysłaniu linku potwierdzającego rejestrację na podany wcześniej adres email
           \end{enumerate}
       \end{enumerate}
       Warunek końcowy: Użytkownik otrzymał nowe konto w aplikacji \\

    \textbf{Logowanie zarejestrowanego użytkownika} \\
        Opis: Użytkownik loguje się do systemu \\
        Aktorzy: Wszyscy, którzy posiadają aktywne konto w aplikacji \\ 
        Warunki wstępne: Aktor nie jest zalogowany, posiada jednak konto na które chce się zalogować \\
        Podstawowy przebieg:
        \begin{enumerate}
            \item Naciśnięcie przycisku do logowania użytkownika
            \item Otworzenie nowego widoku z prośbą o podanie adresu mailowego i nazwy użytkownika
            \item System sprawdza poprawność wprowadzonych danych
            \begin{enumerate}
               \item W przypadku odmowy, system informuje użytkownika o tym zdarzeniu i prosi o poprawne dane użytkownika
               \item W przypadku zatwierdzenia danych, użytkownik zostaje przekierowany do panelu głównego aplikacji
           \end{enumerate}
        \end{enumerate}
        Warunek końcowy: Użytkownik otrzymał się do aplikacji \\

    \textbf{Przechowywanie sesji użytkownika}
        Opis: Użytkownik po wcześniejszym udanym logowaniu do aplikacji bez poprzedzającego wylogowania się z niej i zamknięciu aplikacji z poziomu swojego urządzenia mobilnego chce ją ponownie uruchomić bez konieczności ponownego logowania się \\
        Aktorzy: Wszyscy użytkownicy posiadający aktywne konto w aplikacji \\ 
        Podstawowy przebieg:
        \begin{enumerate}
            \item Użytkownik poprawnie loguje się do aplikacji używając swojego konta użytkownika
            \item Zamknięcie aplikacji bez wylogowania się z niej
            \item Użytkownik ponowne otwiera aplikację na swoim urządzeniu mobilnym
            \item Aplikacja automatycznie przekierowuje użytkownika do panelu głównego do aplikacji
        \end{enumerate}
        Warunek końcowy: Użytkownik poprawnie został przekierowany do ekranu głównego bez konieczności ponownego wpisywania swoich danych \\

    \textbf{Przeglądanie dostępnych atrakcji turystycznych} \\
        Opis: Użytkownik chce zobaczyć jakie dostępne atrakcje turystyczne są dostępne w jego okolicy \\
        Aktorzy: Wszyscy, którzy posiadają aktywne konto w aplikacji \\ 
        Warunki wstępne: Użytkownik w aplikacji chce zobaczyć listę dostępnych zabytków które są położone w niewielkiej odległości od niego \\ 
        Podstawowy przebieg:
        \begin{enumerate}
            \item Użytkownik klika w zakładkę zatytułowaną "Atrakcje"
            \item Aplikacja zwraca listę obiektów turystycznych w kolejności od tych najbliżej lokalizacji użytkownika do tych najdalszych w określonym promieniu
            \item Użytkownik może przeglądać listę takich obiektów i przypadku gdy jakiś go zainteresuje, może kliknąć w niego aby otrzymać szczegółowe informacje na temat danej atrakcji
        \end{enumerate}
        Warunek końcowy: Użytkownik ma możliwość wyświetlania atrakcji turystycznych \\

    \textbf{Wyszukiwanie atrakcji turystycznej} \\
        Opis: Użytkownik chce znaleźć informacje o interesującej go atrakcji turystycznej \\
        Aktorzy: Wszyscy, którzy posiadają aktywne konto w aplikacji \\ 
        Warunki wstępne: Użytkownik zalogowany w aplikacji chce znaleźć informacje o danej atrakcji turystycznej która go interesuje \\ 
        Podstawowy przebieg:
        \begin{enumerate}
            \item Użytkownik klika w zakładkę wyszukiwania obiektów
            \item Użytkownik wpisuje interesująca go atrakcję
            \item Aplikacja zwraca wynik wyszukiwania
            \begin{enumerate}
                \item Jeżeli dany obiekt istnieje, użytkownik klika w zwrócony wynik
                \item Jeżeli danego obiektu nie ma w bazie obiektów, wyszukiwarka nie zwraca żadnego wyniku
            \end{enumerate}
        \end{enumerate}
        Warunek końcowy: Użytkownik wyszukuje atrakcje turystyczne \\

    \textbf{Zaproszenie użytkownika do wspólnego zwiedzania} \\
        Opis: Użytkownik chce odwiedzić atrakcję turystyczną wspólnie z innym użytkownikiem aplikacji \\
        Aktorzy: Wszyscy, którzy posiadają aktywne konto w aplikacji \\
        Warunki wstępne: Użytkownik zalogowany w aplikacji chce znaleźć informacje o pobliskich zabytkach i zaprosić użytkownika znajdującego się w pobliżu \\
        Podstawowy przebieg:
        \begin{enumerate}
            \item Użytkownik klika w zakładkę wyszukiwania obiektów
            \item Użytkownik filtruje obiekty w wyszukiwarce tak aby pokazywały się jedynie te w określonej, niedalekiej odległości od niego
            \item Użytkownik wybiera interesującą go atrakcję turystyczną
            \item Na stronie atrakcji ma dostęp do opcji "Zaproś", którą wybiera
            \item Wybierana jest osoba która ma zostać zaproszona
        \end{enumerate}
        Warunek końcowy: Użytkownik wysyła zaproszenie innemu użytkownikowi \\

    \textbf{Akceptowanie zaproszenia} \\
        Opis: Użytkownik otrzymuje zaproszenie na wspólne zwiedzanie zabytku od innego użytkownika, chce on takie zaproszenie przyjąć \\
        Aktorzy: Wszyscy zalogowani użytkownicy, którzy otrzymali zaproszenie \\
        Podstawowy przebieg:
        \begin{enumerate}
            \item Otrzymanie zaproszenia od innego użytkownika
            \item Sprawdzenie atrakcji turystycznej w danym zaproszeniu
            \item Kliknięcie przycisku "akceptuj"
            \item Użytkownik zapraszający otrzymuje powiadomienie o akceptacji zaproszenia
        \end{enumerate}
        Warunek końcowy: Użytkownik zaakceptował otrzymane zaproszenie od innego użytkownika aplikacji \\

    \textbf{Odmowa zaproszenia} \\ 
        Opis: Użytkownik otrzymuje zaproszenie na wspólne zwiedzanie zabytku od innego użytkownika, chce on odmówić \\
        Aktorzy: Wszyscy zalogowani użytkownicy, którzy otrzymali zaproszenie \\
        Warunki wstępne: Użytkownik zalogowany w aplikacji otrzymał od innego użytkownika zaproszenie do wspólnego zwiedzenia atrakcji turystycznej
        \begin{enumerate}
            \item Otrzymanie zaproszenia od innego użytkownika
            \item Sprawdzenie atrakcji turystycznej w danym zaproszeniu
            \item Kliknięcie przycisku "odmów"
            \item Użytkownik zapraszający otrzymuje powiadomienie o odmowie
        \end{enumerate}
        Warunek końcowy: Użytkownik usunął otrzymane zaproszenie od innego użytkownika aplikacji \\

    \textbf{Dodanie atrakcji turystycznej do listy ulubionych} \\ 
        Opis: Użytkownik dodaje atrakcję turystyczną do listy swoich ulubionych atrakcji \\
        Aktorzy: Wszyscy, którzy posiadają aktywne konto w aplikacji \\
        Warunki wstępne: Zalogowny użytkownik przegląda atrakcje turystyczne \\
        \begin{enumerate}
            \item Użytkownik przegląda listę dostępnych atrakcji turystycznych
            \item Kliknięcie w interesującą pozycję i przejście do ekranu pokazującego szczegóły tej atrakcji
            \item Kliknięcie przycisku "Dodaj do ulubionych"
            \item Użytkownik otrzymuje powiadomienie o dodaniu nowej atrakcji turystycznej do listy ulubionych
        \end{enumerate}
        Warunek końcowy: Użytkownik dodał atrakcję turystyczną do listy swoich ulubionych \\

    \textbf{Usunięcie atrakcji turystycznej z listy ulubionych} \\ 
        Opis: Użytkownik dodaje atrakcję turystyczną do listy swoich ulubionych atrakcji \\
        Aktorzy: Wszyscy, którzy posiadają aktywne konto w aplikacji \\
        Warunki wstępne: Zalogowny użytkownik znajduje się w ekranie głównym aplikacji \\
        \begin{enumerate}
            \item Użytkownik klika przycisk przekierowujący go do ekranu z jego profilem użytkownika
            \item Wybranie pozycji "Ulubione atrakcje"
            \item Przeglądanie listy ulubionych atrakcji
            \item Kliknięcie w pozycję która już go nie interesuje i przejście do ekranu przedstawiającego szczegóły tej atrakcji
            \item Kliknięcie przycisku "Usuń z ulubionych"
            \item Użytkownik otrzymuje powiadomienie o usunięciu atrakcji turystycznej z listy ulubionych
        \end{enumerate}
        Warunek końcowy: Użytkownik usunął atrakcję turystyczną z listy ulubionych \\